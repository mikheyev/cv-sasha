\documentclass[11pt]{article}

% This is a tex file to make Alexander Mikheyev's CV.
% The content is

\usepackage[utf8]{inputenc}

\usepackage{rz-vita}
%\usepackage[english]{isodate}
%\printdayoff\numdate
\def\printdate#1{\xprintdate#1-}
\def\xprintdate#1-#2-#3-{#1}
\def\daterange#1#2{\xprintdate#1---\xprintdate#2-}

% Your biblatex file is likely somewhere else.
\addbibresource{mikheyev.bib}


\title{}
\name{Sasha (Alexander) S. Mikheyev}
% I guess postnoms is like junior? I dunno.
\postnoms{}
\address{
  Australian National University\\
  Research School of Biology\\
  Divisin of Ecology and Evolution\\
  RN Roberson Building\\
  46 Sullivans Creek Rd.\\
  Acton, ACT 2601\\
  Australia\\
}
\www{homologo.us}
\email{sasha@homologo.us}
\tel{+61 (0) 490 485 469}
\twitter{amikheyev}
\orcid{0000--0003-4369--1019}
\github{mikheyev}
\googlescholar{d1Q6iL0AAAAJ}
\subject{}


\begin{document}

\maketitle

\section{Education}

\ind PhD, \textbf{University of Texas, Austin}, Integrative
Biology, \printdate{2009-05-01}


\ind MA, \textbf{University of Florida,
Tallahassee}, Biology, \printdate{2002-05-01}


\ind B.A., \textbf{Cornell University}, Neurobiology and
Behavior, \printdate{2000-05-01}



\section{Appointments}

\subsection{Australian National University, Research School of Biology}
\ind Associate Dean (International) (0.2 FTE). \printdate{2019-00-00}--.

\ind Associate Professor. \printdate{2020-01-01}--.

\ind Senior Lecturer. \daterange{2018-06-01}{2019-31-12}.

\ind Group Leader (Future Fellow). \printdate{2017-07-30}--.

\subsection{Okinawa Institute of Science and Technology}
\ind Associate Professor (adjunct). \daterange{2017-00-00}{2021-00-00}.

\ind Associate Professor. \daterange{2015-00-00}{2017-00-00}.

\ind Assistant Professor. \daterange{2012-00-00}{2015-00-00}.

\ind Independent New Investigator. \daterange{2009-08-00}{2012-00-00}.


\section{Academic awards}

\ind \textit{Future Fellowship (level 2)}, Australian Research
Council, 2016.

\ind \textit{\href{https://news.utexas.edu/2008/05/29/graduate-students-honored-for-excellence}{George
H. Mitchell Award for Excellence in Graduate Research}}, University of
Texas, 2008, (the university's top graduate research award).

\ind \textit{\href{http://iussi.cyberbee.net/wp-content/uploads/2010/04/2007\_Fall\_NAS-IUSSI\_Newsletter.pdf}{George
Eickwort Award}}, North American Section of the International Union for
the Study of Social Insects, 2007, (for ``exceptional research and
scholarly activity by students in the field of social insect biology'').


\section{Editor}

\ind \textit{\href{https://mc.manuscriptcentral.com/jte}{Journal of
Tropical Ecology}}, Editor-in-Chief. \daterange{2019-00-00}{2020-00-00}.

\ind \textit{\href{https://peerj.com/sasha/}{PeerJ}}, Academic
Editor. \printdate{2015-00-00}--.

\ind \textit{\href{http://openlogicproject.org/}{Myrmecological
News}}, Subject Editor. \printdate{2013-00-00}--.


\section{Research Grants \& Fellowships}

\subsection{Grants}

\ind \textit{Ministry of Education, Culture, Sports, Science \&
Technology in Japan}, KAKENHI Grant-in-Aid for Scientific Research
Bilateral Grant (with Israel), ``An integrative 'omics approach to
identify chemosensory proteins in the mouth parts of the honey bee
parasitic \textit{Varroa} mite''.  JPY 5,000,000. 2020 (Project Lead, in
collaboration with V. Soroker (The Volcani Agriculture Institute,
Israel)).


\ind \textit{National Computational Infrastucture}, National
Computational Merit Allocation Scheme, ``Understanding the advantages of
sex, using asexual lineages''.  40k service units (kSU). 2020 (Project
Lead).


\ind \textit{Australian National University}, Vice-Chancellor's Teaching
Enhancement Grant, ``Teaching quantitative skills to biologists using an
interactive virtual environment''.  AU\$ 9,989. 2019 (Project Lead,
Co-applicants: Robert Lanfear, Teresa Neeman, Dan Noble and Eric Stone).



\ind \textit{Australian Research Council}, Discovery
Project, ``Resolving the role of DNA methylation in insect social
evolution''.  AU\$ 380,000. 2018 (Co-investigator with Luke Holman).


\ind \textit{Ministry of Education, Culture, Sports, Science \&
Technology in Japan}, KAKENHI Grant-in-Aid for Scientific Research
(B), ``Origins, spread and evolution of novel honey bee parasites and
diseases''.  ¥16,900,000. 2018.


\ind \textit{Ministry of Education, Culture, Sports, Science \&
Technology in Japan}, KAKENHI Fund for the Promotion of Joint
International Research, ``Using machine vision to understand causes and
consequences of collective behavior in a honey bee
society''.  ¥13,130,000. 2016.


\ind \textit{Ministry of Education, Culture, Sports, Science \&
Technology in Japan}, KAKENHI Young Scientist (A), ``Genetic control of
honeybee dance''.  ¥16,640,000. 2016.


\ind \textit{Ministry of Education, Culture, Sports, Science \&
Technology in Japan}, KAKENHI (S), ``Royal Epigenetics: Molecular basis
of the extended longevity of reproductives in social
insects''.  ¥212,940,000. 2013 (co-PI with Kenji Matsuura (lead), Iuchi
Yoshihito and Masaki Kamakura).


\ind \textit{Ministry of Education, Culture, Sports, Science \&
Technology in Japan}, KAKENHI Young Scientist (B), ``Study of sex
determination mechanisms in
\textit{Wasmannia auropunctata}''.  ¥1,430,000. 2013.


\ind \textit{United States National Science Foundation}, Division Of
Environmental Biology, ``Phylogenetic research on the origin and
evolution of agriculture in ants''.  \$444,988.00. 2010 (Co-Pi with Ted
Schultz (Principal Investigator), Charles Mitter, Ulrich Mueller and
Natasha Mehdiabadi).






\subsection{Fellowships}




\ind \textit{Australian Research Council}, Future Fellowship, ``The
origins, global spread and evolution of novel honey bee
parasites''.  AU\$ 809,000. 2018.









\ind \textit{U.S. Natinal Science Foundation}, Postdoctoral fellowship
in biological informatics (\textit{declined}), ``Massive horizontal gene
transfer in bdelloid rotifers: gathering junk or evolutionary
innovation?''.  2009 (A total of 15 fellowships per year are awarded
country-wide. Declined in order to take up an independent position in
Japan.).


\ind \textit{Fulbright Foundation}, Fellowship to Cameroon and
Gabon, ``Impact of an invasive ant on mainland
Africa''.  2005--2006 (This fellowship has a \textasciitilde15\% funding
rate to Sub-Saharan Africa).


\ind \textit{Environmental Protection Agency}, STAR pre-doctoral
fellowship, ``Worldwide traffic, impact and the evolutionary trajectory
of invasive populations in the little fire ant
\textit{Wasmannia auropunctata}''.  2004--2005 (Median grant application
award rate of 16\% in 2003--2014).



\section{Teaching}

\subsection{Australian National University}
\ind Ecology (BIOL2131), lecturer, 2012 -- 2021. (undergad).

\ind Evolution (BIOL2114), lecturer, 2017 -- 2021. (undergad).

\subsection{Okinawa Institute of Science and Technology Graduate
University}
\ind Biology (B02), sole instructor, 2012--2015. (grad).

\ind Quantitative Evolutionary Comparative Biology
Workshop, organizer, 2011, 2012. (grad).

\subsection{University of Maryland University College}
\ind Introductory Biology Lab (BIOL 102), sole
instructor, 2010. (undergrad).

\ind Introductory Biology (BIOL 101), sole
instructor, 2010. (undergrad).

\subsection{Southwestern University}
\ind Genetics and Evolution (BIO50-122), sole
instructor, 2007. (undergad).


\section{Students}

\subsection{Supervisor}
\ind Carmen
Emborski (PhD), ``\href{https://ttu-ir.tdl.org/bitstream/handle/2346/74394/EMBORSKI-DISSERTATION-2018.pdf}{Transgenerational
effects of ancestral dietary modifications: investigations of progeny
response consistency and modes of transmission},'' 2018

\ind Claire
Morandin (PhD), ``\href{https://pdfs.semanticscholar.org/ce84/0a0475d80c276d36c4e6f6fa7a00d0f3524c.pdf}{To
be or not to be a Queen -- Caste-specific gene expression patterns in
ants},'' 2015

\ind Agneesh Barua (PhD), ``,'' 2021

\subsection{External Examiner}
\ind Thomas Dejaco (PhD, University of Innsbruck), ``Integrative species
delimitation in the alpine jumping‐bristletail genus
\textit{Machilis latreille}, 1832,'' 2014

\subsection{Honours/Masters Committee Member}
\ind Somasundhari Shanmuganadam (Masters, supervisors: Benjamin
Schwessinger /Robyn Hall), ``Uncovering the Hare microbiome,'' 2019

\ind Holly Sargent (Honours, supervisor: Craig Moritz), ``Genetic
erosion of island marsupial populations,'' 2018

\subsection{Undergraduate Research Project Supervision}
\ind Emily Jones (PhB), ``Selection for host phenotype extremes via
manipulation of the symbiotic interactions between rhizobia and the
soybean plant,'' 2021

\ind Zaran Zahid (PhB), ``Using computer vision to determine the
foraging activity of honeybees in a 2-D beehive,'' 2021


\section{Presentations}

\subsection{Invited Departmental Presentations}

\ind ``TBD.'' Viikki Lectures, University of Helsinki, Helsinki,
Finland, \printdate{2021-00-00}.




\ind ``How to design a study using next-generation sequencing
tools.'' Australian Bee Genomics Working Group, 2nd Australian Native
Bee Conference, Brisbane, Australia, \printdate{2019-12-00}.


\ind ``Coevolution while you wait: the arms race between honey bees and
ectoparasitic Varroa mites.'' Melbourne University School of BioSciences
Seminar Series, Melbourne, Australia, \printdate{2019-00-00}.








\ind ``Using historical collections to understand the evolutionary
response of bees to an emergent parasite.'' Macquarie
University, Sydney, Australia, \printdate{2016-00-00}.


\ind ``Evolutionary response by wild honey bees to
\textit{Varroa}.'' Plymouth University, Plymouth, United
Kigdom, \printdate{2016-00-00}.


\ind ``Evolutionary response by wild honey bees to
\textit{Varroa}.'' Plymouth University, Plymouth, United
Kigdom, \printdate{2016-00-00}.


\ind ``Evolutionary response by wild honey bees to
\textit{Varroa}.'' Institute of Science and Technology, Vienna,
Austria, \printdate{2016-00-00}.


\ind ``Evolutionary response by wild honey bees to
\textit{Varroa}.'' Max Planck Institute for Evolutionary Biology, Plön,
Germany, \printdate{2016-00-00}.


\ind ``Evolutionary response by wild honey bees to
\textit{Varroa}.'' Texas A\&M University, College Station, Texas, United
States, \printdate{2016-00-00}.


\ind ``Whole-genome re-sequencing of museum specimens reveals resilience
to disease in a feral population of European honey bees.'' Academia
Sinica, Taipei, Taiwan, \printdate{2016-00-00}.


\ind ``Using historical collection to understand feral honey bee
evolution.'' United States Department of Agriculture, Carl Hayden Bee
Research Center, Tucson, Arizon, United States, \printdate{2015-00-00}.


\ind ``Using population genomics to look at selection and evolution in
social insects.'' University of Pennsylvania, Philadelphia,
Pennsylvania, United States, \printdate{2015-00-00}.


\ind ``When subspecies collide: What can genome-wide signatures of
hybridisation tell us about speciation?.'' Australian National
University, Species delimitation in the age of genomics
Workshop, Canberra, Australia, \printdate{2015-00-00}.


\ind ``Museum samples reveal population genomic changes associated with
a rapid evolutionary response by wild honey bees \textit{Apis mellifera}
to a novel parasite.'' Australian National University, Species
delimitation in the age of genomics Workshop, National University of
Singapore Biology Colloquium, \printdate{2015-00-00}.


\ind ``Working with ancient DNA: tools and insights.'' Kzan' Federal
University, Kazan', Russian Federation, \printdate{2014-00-00}.


\ind ``Working with degraded DNA: laboratory and bioinformatic
approaches.'' CSIRO, Canberra, Australia, \printdate{2014-00-00}.


\ind ``Whole genome re-sequencing of museum specimens revels resilience
to disease in a feral population of European honey bees.'' Australian
National University, Canberra, Australia, \printdate{2014-00-00}.



\ind ``Molecular signatures of ancient mutualistic coevolution in attine
ants and their fungal cultivars.'' Experimental Evolution Discussion
Group, Wageningen University, Wageningen,
Netherlands, \printdate{2014-00-00}.


\ind ``Whole genome re-sequencing of museum specimens revels resilience
to disease in a feral population of European honey bees.'' University of
Southern California, Los Angeles, California, United
States, \printdate{2014-00-00}.


\ind ``Whole genome re-sequencing of museum specimens revels resilience
to disease in a feral population of European honey bees.'' University of
Southern California,, Los Angeles, California, United
States, \printdate{2014-00-00}.


\ind ``Working with degraded DNA: laboratory and bioinformatic
approaches.'' University of Southern California,, Los Angeles,
California, United States, \printdate{2014-00-00}.


\ind ``Development and applications of techniques to work with degraded
DNA.'' Kzan' Federal University, Kazan', Russian
Federation, \printdate{2013-00-00}.



\ind ``Using museum collections to understand how honey bees have
survived a disease pandemic.'' School of Life Sciences, Arizona State
University, Tempe, Arizona, \printdate{2013-00-00}.


\ind ``Using museum collections to understand how honey bees have
survived a disease pandemic.'' School of Life Sciences, Arizona State
University, Tempe, Arizona, \printdate{2013-00-00}.


\ind ``Using museum collections to understand how honey bees have
survived a disease pandemic.'' Integrative Biology, University of
Texas, Austin, Texas, United States, \printdate{2013-00-00}.


\ind ``Using museum collections to understand how honey bees have
survived a disease pandemic.'' Neurobiology and Behavior, Cornell
University, Ithaca, New York, United States, \printdate{2013-00-00}.


\ind ``Evolutionary fate of horizontally acquired genes in bdelloid
rotifers.'' Ehwa Womans University, Seoul,
Korea, \printdate{2013-00-00}.


\ind ``Using museum collections to understand how honey bees have
survived a disease pandemic.'' CAS-MPG Partner Institute for
Computational Biology, Shanghai, China, \printdate{2012-00-00}.





\ind ``The role of adaptation and mutualistic interactions in the
invasion of an African rainforest by the little fire ant
\textit{Wasmannia auropunctata}.'' Rice University, Houston, Texas,
United States, \printdate{2008-00-00}.


\ind ``The spread of an invasive ant in African lowland rainforest:
effects of environment and genetics.'' University of Hawaii, Hilo,
Hawaii, United States, \printdate{2008-00-00}.


\ind ``Coevolution at the phylogenetic vs.~population-genetic scales in
the attine ant-fungal cultivar symbiosis.'' Wageningen
University, Wageningen, Netherlands, \printdate{2007-00-00}.



\subsection{Invited Conference and Workshop Talks}










\ind ``Staying alive: Genetic confirmation that the Lord Howe stick
insect is not extinct.'' \emph{Genomics and collections: adaptation to
macroevolution}, Canberra, Australia, \printdate{2017-06-14}.












\ind ``When subspecies collide: What can genome-wide signatures of
hybridisation tell us about speciation?.'' \emph{Australian National
University, Species delimitation in the age of genomics
Workshop}, Canberra, Australia, \printdate{2015-00-00}.


























\subsection{Oral Conference Presentations}









\ind ``Staying alive: Genetic confirmation that the Lord Howe stick
insect is not extinct.'' \emph{Entomological Society of
America}, Boulder, Colorado, United States, \printdate{2018-00-00}.



\ind ``Using historical collections to understand the evolutionary
response of bees to an emergent parasite.'' \emph{Society for Molecular
Biology and Evolution}, Austin, Texas, United
States, \printdate{2016-00-00}.





























\ind ``Transcriptional profiling of attine ant fungal
symbionts.'' \emph{British Mycological Society}, Alicante,
Spain, \printdate{2012-00-00}.


\ind ``Invasion of lowland rainforest by their little fire ant
\textit{Wasmannia auropunctata}.'' \emph{Department and Graduate
Institute of Entomology, National Taiwan University}, Taipei,
Taiwan, \printdate{2011-00-00}.







\subsection{Poster Conference Presentations}
































\ind ``RAD-tagging and low-coverage shotgun phylogenetics with degraded
DNA from non-destructively sampled museum
specimens.'' \emph{Illumina}, Phuket, Thailand, \printdate{2013-00-00}.















\subsection{Local Talks}


\ind ``The microbiome wants what it wants: microbial evolution overtakes
experimental host-mediated indirect selection.'' \emph{Australian
National University, Research School of Biology Faculty
Flash}, Canberra, Australia, \printdate{2019-06-26}.


\ind ``The microbiome wants what it wants: microbial evolution overtakes
experimental host-mediated indirect selection.'' \emph{Australian
National University, Research School of Biology Faculty
Flash}, Canberra, Australia, \printdate{2019-06-26}.
























\ind ``Molecular signatures of ancient mutualistic coevolution in attine
ants and their fungal cultivars.'' \emph{OIST-NTU Invasive Ant
Symposium, Okinawa Institute of Science and Technology}, Okinawa,
Japan, \printdate{2014-00-00}.
















\ind ``Ecology and evolution: the next synthesis.'' \emph{Sydney Brenner
commemorative symposium, Okinawa Institute of Science and
Technology}, Okinawa, Japan, \printdate{2011-00-00}.






%\filbreak  % may not be necessary in the future, prevents heading from being leftover
\defbibheading{publist}[\bibname]{%
  \subsection{#1}}

\begin{publications}
  \printbibrz{article}{Journal Articles}
  \printbibrz{conferences}{Conference papers \& posters}
  \printbibrz{commentary}{Commentary}
\end{publications}

\section{Service}

\subsection{Associate Dean (International), Joint Colleges of Science,
Health and Medicine, Australian National University}
\ind Indian Institute of Technology Madras Joint Degree Program steering
committee. 2019--.

\ind Joint management committee for the Australian National
University/Shandong University Joint Science College. 2019--.

\subsection{Senior Lecturer, Research School of Biology, Australian
National University}
\ind . 

\subsection{Professional Organizations}
\ind Australian Entomological Society, Education Committee
member. 2018--.

\subsection{Reviewer}
\ind \textbf{Journals}: Applied Sciences, Bioinformatics, Biology
Letters, BMC Biology, BMC Evolutionary Biology, BMC Genomics,
Communications Biology, Current Opinion in Insect Science, Ecological
Research, Ecology, Ecology and Evolution, Ecology Letters, Insectes
Sociaux, Journal of Animal Ecology, Journal of Evolutionary Biology,
Journal of Visualized Experiments, Molecular Biology and Evolution,
Molecular Ecology, Molecular Ecology Resources, Nature Ecology \&
Evolution, Philosophical Transactions of the Royal Society B: Biological
Sciences, PLoS Computational Biology (Guest Editor), PLoS Genetics,
Proceedings of the Royal Society B: Biological Sciences, Scientific
Data, Scientific Reports, Toxins. 

\ind \textbf{Granting Agencies}: Australian Research Concil, Deutsche
Forschungsgemeinschaft, Israeli Ministry of Science and Technology
(MOST), U.S. National Science Foundation. 


\section{Outreach}

\subsection{TV Interview}
\ind SBS
News.  \href{https://www.sbs.com.au/news/australian-beekeepers-call-for-ban-on-potentially-harmful-pesticide}{Commenting
on the use of neonicotinoid pesticides}. 2019-05-08

\subsection{Public talk}
\ind Japan Institute, ANU College of Asia \& the Pacific. OBJECTively -
Connecting Australia and Japan: objects, cultural stories,
people, \href{http://japaninstitute.anu.edu.au/events/objectively-connecting-australia-and-japan-objects-cultural-stories-people}{Coversations:
bee parasitc mites}. 2019-03-08

\ind Australia - Japan Society (ACT). OBJECTively - Connecting Australia
and Japan: objects, cultural stories, people, Bee line between Okinawa
and Canberra. 2019-09-28


\end{document}
